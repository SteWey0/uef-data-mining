This course gave insights to quite a few topics. 
We started by talking about statistics and Data Mining in general. 
In my opinion, the most important concepts here where estimators, covariance and correlation. 
Next we learned about handling missing data and how to quantify a "distance" between rows of a dataframe by using metrics. 
We discussed the importance of scaling the data entries and how that could affect the resulting models. 
All these steps we learned about in the first two weeks were practiced in the corresponding exercises, but also came up as vital pre-processing steps in all other exercises that followed. 
The material then continued with visualizing data by using e.g. box plots or Sammon mapping and dealing with time series by using periodograms or filters like moving averages. 
This logically expanded the topics we discussed prior, was practiced in the exercise and used for every following exercise. 
After talking about how to visualize data, compression was the logical next step to gaining insights or models. 
As such, we discussed clustering data via 3 different kinds of algorithms, compared how they work and for which kind of data they are suited. 
Logically, we then also talked about how to validate how well the clustering worked. 
Lastly, the final lecture built on all the concepts learned previously and dealt with predictive modelling by using linear regression or machine learning. 
To find the quality of the resulting model, different performance metrics have been presented. 
The last exercises took us through the whole data processing chain to obtain predictive models in the end, which were then compared. \\

As can be seen from this short summary, the lecture contents of each week built very logically on top of each other. 
We learned about the process of Data Mining from the ground up, starting with loading data, handling missing values, doing explorative data analysis, visualizing the data, compressing it via clustering and lastly formulating models that describe it. 
The exercises similarly followed this procedure. \\

For me and my future studies it will prove quite helpful to have such a well-structured learning diary and also exercises with code snippets for every step we talked about. 
I mentioned many times already, that many things that were taught in this course were already known to me beforehand. 
However, every topic offered something new, that expanded this knowledge and writing the diary lead to many new connections and insights that I am sure will be of use for my future studies. 
The main thing that remains unclear after having done this course is the practical implementation of the methods we learned. 
The code in the exercises was mainly already finished and as such, I could not do the specific implementation of e.g. an ARX model. 
However, it is still nice to know the basics behind the techniques and having these well-structured code snippets for all the different topics through the exercises. 
I am sure, when the need arises, I would be able to formulate models based on the learning diary, the code snippets and some additional research. 