All in all, I think I am rather happy with how I proceeded in this course. 
I managed to stick roughly to the schedule and do something for the course every week. 
I attended this course completely remotely, as I am in Joensuu campus and therefore typically watched the lecture videos and summarized them before the weekly test and did the exercises after that. 
Then I proceeded to writing the learning diary by including all that I learned in the course and by trying to think how that fits into what I've learned so far. 
During the first few weeks this personal style of writing still felt rather odd to me as I was not used to it. 
However, in the following weeks I have grown more and more used to it and managed to write the diary faster. 
This progression of learning to write a diary and not a strict report also manifested itself in the summaries themselves. 
In the beginning I rather felt like I needed to cover everything in the lecture and had a hard time deciding what and how to write. 
It was hard to be precise and cover the whole lecture while still being concise and having the summary roughly two pages long. 
As the weeks passed, I felt that my writing style became more free and that I managed to focus more on what actually interested me and what was new to me. 
I realized that it was okay to leave things out and give more of a qualitative overview, to focus on the things that stuck with me or that were especially useful during the exercises. 
This learning process also led to me being a bit more efficient when writing. 
Looking back on this course, I think this integrated approach of a learning diary really helped my understanding by needing to put all new things into context and thinking about how they relate to me personally. 
As such it was quite an interesting experience. \\

Another big contributor to my efficiency was the use of AI. 
In the beginning, I hardly used AI as I did not feel to comfortable with it and rather wanted to think through the topics and how I want to structure the summary myself. 
As the course progressed, I have however found a way to compromise in order to still having to think and write for myself but also letting AI take some workload in certain areas. 
Some areas where I remained adamant against using AI still remained. 
These were mainly:
\begin{itemize}
    \item Writing this document. I feel like it is more work for me to let a LLM write text with which I am happy and that is coherent in terms of the 20 pages or so that were written before, than it is to write the text myself. Additionally, writing summaries is one of the main ways I understand topics, and it is a great way to practice my written English. As such, I did not want to pass this important learning opportunity on to a LLM.
    \item Working through the lectures and thinking about connections to my prior studies. I really enjoyed this aspect a lot about this course. It was great fun to be able to find so many similarities and connections to prior courses, and it also helped my understanding quite a lot. As such I did not want to use AI to work through the lectures instead of me. 
    \item Doing the exercises. This course exercises focused mainly on understanding the impact of different methods and less on coding itself. In my opinion, this approach is only beneficial, if one thinks themselves about the questions posed.
\end{itemize}
However, there were also some use-cases were AI has proven itself to be a useful tool. 
These included:
\begin{itemize}
    \item Summarizing lectures and finding additional connections between them or topics I learned before that were not covered in the lecture. It was quite helpful to quickly be able to confirm connections between concept A from the lecture and concept B that I still vaguely remembered from previous courses. Additionally, at times it was helpful to pose questions or see different formulations of the topics covered in this course. 
    \item Code questions. Be it plotting with \lstinline{Matplotlib} in some certain style I imagined or needing help with \lstinline|pandas|, it was quite helpful to use a LLM to get some quick insights.
    \item LaTeX layout. Sometimes TeX can be quite a hassle, especially when the figures or tables become more complex. AI has proven itself useful to take some workload there and let me focus on the more important parts. 
    \item Result synthesis. Especially the last two exercises had a lot of repetitive cell outputs, that needed to be transformed to tables. I found a great workflow that saved an immense amount of time, by letting a LLM transform these outputs to Markdown tables, and after that select certain data to form the final LaTeX table. Of course, still every value needed to be checked to exclude hallucination, but this was way faster than doing this repetitive task myself.  
\end{itemize}
All in all I would therefore say, that I changed my learning techniques quite substantially during the lecture. 
This lead to a much more efficient workflow, where I was able to focus more on understanding lectures and connecting them with other knowledge. \\

If I were to do the lecture again I would start earlier with experimenting on using AI to help me be more productive. 
Additionally, I would try to do more work at the beginning of the lecture, in order to have less of a tight schedule at the end. 
Still, I am quite happy with how things have worked out in the end. 
I have reached my learning goals both in terms of the knowledge that was taught during the course, but also in learning how to work through and summarize the material efficiently using AI and putting a larger focus on the connections to prior knowledge. 
