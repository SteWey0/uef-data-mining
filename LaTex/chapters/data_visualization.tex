Sammons mapping:
\begin{itemize}
    \item Sammon mapping weights small distances more strongly, so local neighborhoods matter the most.
    \item Given $d\in R^D$ we calculate all pairwise distances $d_{ij}=\|x_i-x_j\|$ 
    \item to achieve a mapping $y\in R^d$ where $d<D$, usually 2, we calculate distances again $d'_{ij}=\|y_i-y_j\|$
    \item then minimize the cost function $E=\frac{1}{\sum_{i<j}d_{ij}}\sum_{i<j}\frac{(d_{ij}-d'_{ij})^2}{d_{ij}}$. Note that this weighs small distances more (denominator) and the first factor is just normalization
    \item updates through gradient descent $y_i\leftarrow y_i-\eta\frac{\partial E}{\partial y_i}$
    \item Optimization is slow (O(N²) distance computations).
    \item Sensitive to local minima results depend on initialization.
    \item Less powerful than modern methods like t-SNE
    \item Preserves distances, especially well small ones
\end{itemize}

Self organizing maps:
\begin{itemize}
    \item Unsupervised Neural network of nerons on a (usually) 2d grid, each of them has a weight vector $w\in R^D$ when the input is $x\in R^D$
    \item show many samples, for each of them we calculate the best matching unit (BMU) i.e. the closest neuron $\text{BMU}(x)=\arg\min_i\|x-w_i\|$
    \item The BMU and its neighbors are updated to be more close to the input. The update is based on the distance of the neuron to the BMU 
    $w_i(t+1)=w_i(t)+\alpha(t)\,h_{ci}(t)\,(x-w_i(t))$ with the neighborhod function (Gaussian)
    $h_{ci}(t)=\exp\!\left(-\frac{\|r_c-r_i\|^2}{2\sigma(t)^2}\right)$
    \item This way Nearby neurons on the grid end up representing similar types of inputs.
    \item So the grid becomes a kind of flattened, simplified version of the input space.
    \item If two neurons are adjacent on the grid → their prototypes are similar in high-D space.
    \item Clusters appear as contiguous regions.
    \item Boundaries between clusters appear as lines or curves on the map.
    \item Analogy: A SOM grid acts like a rubber sheet with fixed positions (the neuron grid). This sheet is pulled toward data points, but it stays smooth because neurons are connected. The 2D sheet gives a flattened but structurally consistent view of the high-D data.
    \item an important visulaization is the U-matrix $U(i)=\frac{1}{|N(i)|}\sum_{j\in N(i)}\|w_i-w_j\|$ which contains average distances between each neuron and its neighbors weights (N is amount of neighbors)
    \item if these are small values are simmilar, if large quite different
    \item plot as heatmap to see clusters and cluster boundaries
\end{itemize}


This week's lecture focussed on visualizing high dimensional data and working with data in time series. 
Especially the first part contained many new things to me, which is why I will put a bigger emphasis on it in the following paragraphs. 
If data is low dimensional, visualization via histograms, box plots, scatter plots, or other methods is straightforward and has been discussed in the chapters above already. 
A more interesting question is how very high dimensional data with maybe hundreds of entries (columns in a dataframe) can be visualized in a 2-dimensional space to get a grasp on clusters and for example prepare the data for clustering algorithms to obtain data classes. 
We discussed to examples of dimensionality reduction here, namely Sammon mapping and self organizing maps (SOMs). 
These are quite interesting to me, as I only knew about Principal component analysis (which has the disadvantage of being linear) t-SNE prior to this lecture. 
Both Sammon mapping and SOMs are nonlinear and can therefore capture the usually complex high dimensional manifolds spanned by the data but are still easy to understand. \\

Sammon mapping is based on the idea of pairwise conserving distances in the mapping of input vectors $x\in \mathbb{R}^D$ to output vectors $y\in \mathbb{R}^d$ with $d<D$ (usually $d=2$). 
Specifically, the algorithm is constructed such, that small distances are weighted more strongly. 
This means, it tries to achieve a mapping that preserves local neighborhoods well and therefore preserves grouping of the data points. 
Sammon mapping follows these steps:
\begin{enumerate}
    \item First, pairwise distances between all inputs $d_{ij}=\|x_i-x_j\|$ and all outputs $d'_{ij}=\|y_i-y_j\|$ are calculated. For the first iteration outputs can be initialized randomly. 
    \item A cost function is calculated as $E=\frac{1}{\sum_{i<j}d_{ij}}\sum_{i<j}\frac{(d_{ij}-d'_{ij})^2}{d_{ij}}$. Note how the denominator leads to a larger weight for small distances. The prefactor is just for normalization.
    \item This cost function is minimized through usual gradient descent and the output mappings are updated as $y_i\leftarrow y_i-\eta\frac{\partial E}{\partial y_i}$, where $\eta$ is a preset learning rate. 
\end{enumerate} \todo{check equatoins. they differ from lecture}
The downsides of Sammon mapping include its sensitivity to local minima leading to failed mappings (due to random initialization), and that the algorithm is rather slow ($\mathcal{O}(N^2)$ distance calculations). 
However, it is useful if distances should be preserved well. \\

Self organizing maps are an unsupervised neural network, where neurons are placed on a usually fixed, 2-dimensional grid. 
Each neuron has a weight vector $w\in \mathbb{R}^D$ when the data is given by $x\in \mathbb{R}^D$. 
The main idea is to find for each input sample (row of the dataset) the closest neuron, i.e. the one where the weight matches closely the input value itself. 
This neuron and its surrounding are then updated to be more similar to the input. 
After learning one tries to achieve the situation where each neuron reflects one "typical" input sample with neighboring neurons in the 2-dimensional grid being similar. 
This way a high dimensional input space is represented in a flattened, although topologically similar space. 
The following steps are important:
\begin{enumerate}
    \item First, the best-matching unit (BMU), the closest neuron, is calculated: $\text{BMU}(x)=\arg\min_i\|x-w_i\|$
    \item The BMU and its neighbors are updated to be more similar to the input. The update is based on the distance of the neuron to the BMU: 
    $w_i(t+1)=w_i(t)+\alpha(t)\,h_{ci}(t)\,(x-w_i(t))$ with the neighborhod function (Gaussian)
    $h_{ci}(t)=\exp\!\left(-\frac{\|r_c-r_i\|^2}{2\sigma(t)^2}\right)$. Here $\alpha(t)$ is the learning rate, 
\end{enumerate}
\todo{add boldface vectors here and before}