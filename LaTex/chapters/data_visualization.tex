Sammons mapping:
\begin{itemize}
  \item Sammon mapping weights small distances more strongly, so local neighborhoods matter the most.
  \item Given $d\in R^D$ we calculate all pairwise distances $d_{ij}=\|x_i-x_j\|$ 
  \item to achieve a mapping $y\in R^d$ where $d<D$, usually 2, we calculate distances again $d'_{ij}=\|y_i-y_j\|$
  \item then minimize the cost function $E=\frac{1}{\sum_{i<j}d_{ij}}\sum_{i<j}\frac{(d_{ij}-d'_{ij})^2}{d_{ij}}$. Note that this weighs small distances more (denominator) and the first factor is just normalization
  \item updates through gradient descent $y_i\leftarrow y_i-\eta\frac{\partial E}{\partial y_i}$
  \item Optimization is slow (O(N²) distance computations).
  \item Sensitive to local minima results depend on initialization.
  \item Less powerful than modern methods like t-SNE
  \item Preserves distances, especially well small ones
\end{itemize}

Self organizing maps:
\begin{itemize}
  \item Unsupervised Neural network of nerons on a (usually) 2d grid, each of them has a weight vector $w\in R^D$ when the input is $x\in R^D$
  \item show many samples, for each of them we calculate the best matching unit (BMU) i.e. the closest neuron $\text{BMU}(x)=\arg\min_i\|x-w_i\|$
  \item The BMU and its neighbors are updated to be more close to the input. The update is based on the distance of the neuron to the BMU 
  $w_i(t+1)=w_i(t)+\alpha(t)\,h_{ci}(t)\,(x-w_i(t))$ with the neighborhod function (Gaussian)
  $h_{ci}(t)=\exp\!\left(-\frac{\|r_c-r_i\|^2}{2\sigma(t)^2}\right)$
  \item This way Nearby neurons on the grid end up representing similar types of inputs.
  \item So the grid becomes a kind of flattened, simplified version of the input space.
  \item If two neurons are adjacent on the grid → their prototypes are similar in high-D space.
  \item Clusters appear as contiguous regions.
  \item Boundaries between clusters appear as lines or curves on the map.
  \item Analogy: A SOM grid acts like a rubber sheet with fixed positions (the neuron grid).
This sheet is pulled toward data points, but it stays smooth because neurons are connected. The 2D sheet gives a flattened but structurally consistent view of the high-D data.
\item an important visulaization is the U-matrix $U(i)=\frac{1}{|N(i)|}\sum_{j\in N(i)}\|w_i-w_j\|$ which contains average distances between each neuron and its neighbors weights (N is amount of neighbors)
\item if these are small values are simmilar, if large quite different
\item plot as heatmap to see clusters and cluster boundaries
\end{itemize}
