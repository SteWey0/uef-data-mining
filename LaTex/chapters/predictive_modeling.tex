In this weeks lecture we discussed validation for regression models or classifiers, performance indicators (both visually and numerically), linear regression and the basics behind neural networks (NNs), especially mulitlayer perceptrons (MLPs). 
As all topics have been a large part in prior courses I took, I will focus on giving a rough overview and highlighting approaches that are new to me and especially interesting. \\

We started by contrasting different validation approaches which is very important, as evaluating any model on data that has been used in \emph{constructing} the model usually leads to a much too optimistic performance. 
This is true both for machine learning task (where I were quite aware that validation sets are important) but also when doing simple regression. 
% Splitting vlaidation sets when fitting to obtain a measure how good the fit generalised is not something I needed to do prior. 
% This approach is therefore quite interesting for me. 
Splitting validation data from the training set can be done in different ways. 
For example one can perform hold-out validation, which means simply splitting the dataset into training and testing sets. 
The training set is used for constructing the model, while the testing set is used purely for evaluation. 
This approach is simple and fast, but the model performance is usually underestimated as the data used was not used for training. 
Another approach is k-fold cross-validation, where data is split into $k$ clusters. 
Training is done on $k-1$ clusters, the remaining one is used for validation. 
Then the the validation cluster is cycled through the whole dataset, such that in the end every cluster has been used for training and validation. 
The overall performance is then the mean of all cluster performances. 
As a special case of k-fold cross validation one can also divide the data in as many clusters, as there are data points. 
This is then called leave-one-out cross-validation. 
It is numerically more expensive and usually provides an estimate that is too optimistic, but may be the best approach when the data sets are small. \\

We then went on to talk about performance indicators to evaluate the models we construct. 
A very simple approach is to use graphical methods that compare the model output with the observations themselve. 
This can be done using scatter plots or simple line graphs where the model and data are plotted on top of each other. 
I additionally used an approach extensively for lab courses I did in the past which was not covered in the material. 
When using regression, one can look at fit residuals, meaning the difference between the fit and data value. 
If a fit describes the data well, these residuals should be centred around zero and have a spread described by a chi-squared distribution. 
If one instead wants to evalute the performance of a model numerically, one could look at the mean error, mean squared error or root mean squared error. 
These are shown in \autoref{eq:me, mae, rmse} from left to right. 
\begin{align}
    \mathrm{ME} &= \frac{1}{n} \sum_{i=1}^n (y_i - \hat{y}_i) & \mathrm{MAE} &= \frac{1}{n} \sum_{i=1}^n |y_i - \hat{y}_i| & \mathrm{RMSE} &= \sqrt{\frac{1}{n} \sum_{i=1}^n (y_i - \hat{y}_i)^2}
    \label{eq:me, mae, rmse}
\end{align}
Here $y_i$ is a data point, $\hat{y}_i$ the model prediction and $n$ the total amount of data points. 
They all are of the same unit as the data itself and are therefore well interpretable. 
One can use the mean error to find the averaged signal deviation together with its direction, meaning one can find whether the model over- or underestimates the data. 
The MAE omits the directionality. 
Most often the RMSE is used however, as the squaring of the deviation has the nice property that large deviations contribute more to the error. 
One can also decompose the RMSE into statistical and systematic components. 
Other approaches we discussed where the coefficient of determination or $R^2$ value, and the index of agreement (IOA). 
The index of agreement is widely used, as it is unitless and lies between 0 and 1. 
It therefore allows for and easy comparison between different models. 
An index of agreement of 1 indicates a perfect model and one below $0.4$ indicates worse performance than random guessing. 
It is best practise to look at the IOA and RMSE together to find out if a model performs suitably well. 
